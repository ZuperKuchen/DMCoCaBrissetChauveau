
                \documentclass[12pt,a4paper]{article}
\usepackage{natbib}         % Pour la bibliographie
\usepackage{url}            % Pour citer les adresses web
\usepackage[T1]{fontenc}    % Encodage des accents
\usepackage[utf8]{inputenc} % Lui aussi
\usepackage[frenchb]{babel} % Pour la traduction française
\usepackage{numprint}       % Histoire que les chiffres soient bien

\usepackage{amsmath}        % La base pour les maths
\usepackage{mathrsfs}       % Quelques symboles supplémentaires
\usepackage{amssymb}        % encore des symboles.
\usepackage{amsfonts}       % Des fontes, eg pour \mathbb.

\usepackage[svgnames]{xcolor} % De la couleur
\usepackage{geometry}       % Gérer correctement la taille

%%% Si jamais vous voulez changer de police: décommentez les trois 
%\usepackage{tgpagella}
%\usepackage{tgadventor}
%\usepackage{inconsolata}

%%% Pour L'utilisation de Python
\usepackage{pythontex}

\usepackage{graphicx} % inclusion des graphiques
\usepackage{wrapfig}  % Dessins dans le texte.

\usepackage{tikz}     % Un package pour les dessins (utilisé pour l'environnement {code})
\usepackage[framemethod=TikZ]{mdframed}
% Un environnement pour bien présenter le code informatique
\newenvironment{code}{%
\begin{mdframed}[linecolor=Green,innerrightmargin=30pt,innerleftmargin=30pt,
backgroundcolor=Black!5,
skipabove=10pt,skipbelow=10pt,roundcorner=5pt,
splitbottomskip=6pt,splittopskip=12pt]
}{%
\end{mdframed}
}

% Mettez votre titre et votre nom ci-après
\title{Rapport DM \\
Complexité \& Calculabilité}
\author{BRISSET Rémi \& CHAUVEAU Pierre}
%% À décommenter si vous ne voulez pas que la date apparaisse explicitement
\date{\today}

% Un raccourci pour composer les unités correctement (en droit)
% Exemple: $v = 10\U{m.s^{-1}}$
\newcommand{\U}[1]{~\mathrm{#1}}

% Pour discuter avec le prof dans le document: le premier argument est 
% le nom de celui qui fait la remarque et le second la remarque 
% proprement dite: \question{jj}{Que voulez-vous dire par là ?}
% \reponse{Droopy}{I'm very happy...}
\usepackage{todonotes}
\usepackage{comment}
\newcommand{\question}[2]{\todo[inline,author=#1]{#2}}
\newcommand{\reponse}[2]{\todo[inline,color=green,author=#1]{#2}}

% Les guillemets \ofg{par exemple}
\newcommand{\ofg}[1]{\og{}#1\fg{}}
% Le d des dérivées doit être droit: \frac{\dd x}{\dd t}
\newcommand{\dd}{\text{d}}


% NB: le script TeXcount permet de compter les mots utilisés dans chaque section d'un document LaTeX. Vous en trouverez une version en ligne à l'adresse
% http://app.uio.no/ifi/texcount/online.php
% Il suffit d'y copier l'ensemble du présent document (via Ctrl-A/Ctrl-C puis Ctrl-V dans la fenêtre idoine) pour obtenir le récapitulatif tout en bas de la page qui s'ouvre alors.

% Pour récupérer les bonnes entrées bibliographiques, je vous conseille l'usage de scholar.google.fr pour les recherches
% et la récupération des entrée BibTeX comme décrit dans cette vidéo: https://www.youtube.com/watch?v=X-9T2Oaj-5A

% Quelques macros utiles
% La dérivée temporelle, tellement courante en physique, avec les d droits
\newcommand{\ddt}[1]{\frac{\dd #1}{\dd t}}
% Des parenthèses, crochets et accolades qui s'adaptent automatiquement à la taille de ce qu'il y a dedans
\newcommand{\pa}[1]{\left(#1\right)}
\newcommand{\pac}[1]{\left[#1\right]}
\newcommand{\paa}[1]{\left\{#1\right\}}
% Un raccourci pour écrire une constante
\newcommand{\cte}{\text{C}^{\text{te}}}
% Pour faire des indices en mode texte (comme les énergie potentielles)
\newcommand{\e}[1]{_{\text{#1}}}
% Le produit vectoriel a un nom bizarre:
\newcommand{\vectoriel}{\wedge}

\begin{document}

\maketitle
\tableofcontents
\newpage

\section{Présentation du problème}
Un arbre couvrant d'un graphe non orienté $G =\{V,E\}$ est un arbre constitué uniquement d'arête $E(G)$ et contenant tous les sommet $V(G)$. Nos  arbres couvrants seront toujours enracinés: un sommet est distingué et appelé la racine de l’arbre. La profondeur $d(v)$ d’un sommet dans l’arbre est la distance par rapport à la racine, c’est-à-dire le nombre d’arêtes sur le chemin direct de la racine à $v$. La hauteur d’un arbre est la profondeur maximale d’un de ses sommets , $max_{v\in V}(d(v))$. Formellement, le problème HAC est le suivant: \\
\indent \textbf{Entrée : } Un graphe non-orienté $G$ et un entier $k$. \\
\indent \textbf{Sortie : }Existe-t-il un arbre couvrant de $G$ de hauteur exactement $k$ ?\\

\section{HAC est NP-difficile}
\subsection{Réduction \textit{Chemin hamiltonien} vers HAC}

Une reduction peut être de prendre le début du chemin hamiltonien comme racine de l'arbre couvrant puis d'ajouter successivement les sommets du chemin à la suite dans l'arbre de sorte à obtenir un arbre ''en ligne'' (ie de hauteur égale à son nombre de noeuds -1 )
\subsection{Conclusion}

Chemin hamiltonien étant NP-Complet, on peut en conclure que HAC est NP-Difficile.

\newpage
\section{Réduction de HAC vers SAT}
Pour chaque sommet $v \in V$ , il y a k+1 variables $x_{v,h}$
Nous avons choisi de représenter les variables dans une matrice de |V| lignes et de k+1 colonnes ou $x_{v,h}$ est à l'intersection de la v-ième ligne et de la h-ième colonne.
\\

\textbf{1. Pour chaque sommet $v \in V$, il y un unique entier h t.q. $x_{v,h}$ est vrai}

L'idée est qu'il faut que pour chaque ligne exactement une variable soit vrai. Pour chaque sommet, on ecrit une premiere clause en s'assurant qu'il en existe au moins une, puis on teste deux à deux les variables pour s'assurer qu'il en existe excactement une. On a donc:
\[
\bigwedge_{v \in V}( \bigvee_{h = 0}^{k} (x_{v,h}) \wedge \bigwedge_{u \in V, i=0, j=0, i!=j}^{k} \overline{x_{v,i}} \vee \overline{x_{u,j}})
\]
\\

\textbf{2. Il y a un unique sommet $v$ t.q. $d(v) = 0$ (``v est la racine'')}

Même principe que la règle 1 mais pour la 1ere colonne (ie pour toutes les variables du type $x_{v,0}$). On a donc:
\[
\bigvee_{v \in V} x_{v,0} \wedge \bigwedge_{i,j \in V, i!=j} (\overline{x_{i,0}} \vee \overline{x_{j,0}})
\]

\textbf{3. Il y a au moins un sommet $v$ t.q. $d(v)$ = $k$}

Trivial, on verifie qu'au moins une variable de type $x_{v,k}$ est vrai. On a donc:
\[
\bigvee_{v \in V}x_{v,k}
\]

 
\textbf{4. Pour chaque sommet $v$, si $d(v) > 0$, alors il existe un sommet $u$ tel que $uv \in E$ et $d(v) = d(v) - 1$}

\[
\bigwedge_{v \in V, i=0}^{k} \overline{x_{v,i}} \vee \bigvee_{uv \in E} x_{u,i-1}
\]


\end{document}
